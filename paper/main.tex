\documentclass[12 pt]{article}
\usepackage[margin=1in]{geometry}

% Refs
\usepackage[style=nature, backend=bibtex]{biblatex}
\addbibresource{main.bib}
\usepackage[usenames,dvipsnames,svgnames,table]{xcolor}

% Math
\usepackage{amsmath}
\usepackage{amssymb}

% Figures
\usepackage{graphicx}
\newcommand{\figref}[1]{Fig.~\ref{fig:#1}}
\graphicspath{{../plots/}}

% subfigures
\usepackage{caption}
\usepackage{subcaption}

% line spacing
\usepackage{setspace}
\singlespacing

% author note
\usepackage{xcolor}
\newcommand{\note}[1]{\textcolor{red}{[#1]}}

\title{Luis Paper}
\author{Pat Wall}
\date{\today}

\begin{document}

\maketitle

\begin{abstract}
    Heeeyyyyyy this is the abstract baby. ill write this last probably!!
\end{abstract}

\section{Introduction}

\section{Results - Outline}
\begin{enumerate}
    \item bsyn and ke show similar relationships to system dynamics 
    ($\rho=0.693, \; p<0.05$ Fig. \ref{fig:Dynamics}B) and 
    ($\rho=0.511, \; p<0.05$ Fig. \ref{fig:dynamics}C) [figure in supplement?]
    \item Output entropy and dynamics?? \figref{dynamics}A [get statistics?]
    \item $k_e$ and $B_{syn}$ correlate. ($\rho=0.5210, \; p < 0.05$). Figure \ref{fig:ke-bsyn}
    \item Synergy bias is a better predictor of effective connectivity than
          other information theoretic approaches for the overall system. [get stats]
          Fig. \ref{fig:ke-oinfo}
    \item At the per-input level traditional methods also fail to detect redundancy \ref{fig:r-te}
\end{enumerate}

\section{Results - Text}

\subsection{Several measures correlate with complexity of dynamics}

We measure the complexity of the dynamics of a particular CA rule by adding the
transient length $l$ and the period $T$. As this sum increases we say the
dynamics are more complex. This choice allows us to fill in very long dynamics
with a lower bound (see Methods \ref{sec:finding-trans}). We find that the
output entropy of the rule table increases with this sum as well as its parts 
shown in Figure \ref{fig:dynamics}A. 

\begin{figure}
    \includegraphics[width=\textwidth]{k5/all_dynamics.pdf}
    \caption{\textbf{A} Transient length, attractor period, and their sum all 
    correlate with the entropy of the oiutput distribution. \textbf{B} Synergy 
    bias $B_{syn}$ correlates with the the sum of the attractor period and 
    transient length ($\rho=0.693, \; p<0.05$). \textbf{C} Effective
    connectivity, $k_e^*$ also correlates with the sum of atttractor period and
    transient length ($\rho=0.511, \; p<0.05$) and appears qualitatively quite
    similar to $B_{syn}$ in this regard.}
    \label{fig:dynamics}
\end{figure} 

\begin{figure}
    \includegraphics{k5/ke_vs_bsyn_dyn.pdf}
    \caption{Synergy increases with effective connectivity}
    \label{fig:ke-bsyn}
\end{figure}

\begin{figure}
    \includegraphics{k5/ke_o_info_reg.pdf}
    \caption{$k_e^*$ correlates negatively with O-information}
    \label{fig:ke-oinfo}
\end{figure}

\begin{figure}
    \includegraphics{k5/directed_cana.pdf}
    \caption{Transfer entropy fails to identify input redundancy at the per input level}
    \label{fig:r-te}
\end{figure}

\begin{figure}
    \includegraphics{k5/synergy_bias.png}
\end{figure}

\section{Discussion}

\end{document}