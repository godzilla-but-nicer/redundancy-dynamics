\documentclass[12 pt]{article}
\usepackage[margin=1in]{geometry}

% Refs
\usepackage[style=nature, backend=bibtex]{biblatex}
\addbibresource{main.bib}
\usepackage[usenames,dvipsnames,svgnames,table]{xcolor}

% Math
\usepackage{amsmath}
\usepackage{amssymb}

% Figures
\usepackage{graphicx}
\newcommand{\figref}[1]{Fig.~\ref{fig:#1}}
\graphicspath{{../plots/}}

% subfigures
\usepackage{caption}
\usepackage{subcaption}

% line spacing
\usepackage{setspace}
\singlespacing

% author note
\usepackage{xcolor}
\newcommand{\note}[1]{\textcolor{red}{[#1]}}

\title{Luis Paper Figures}
\author{Pat Wall}
\date{\today}

\begin{document}

\maketitle

\begin{figure}
    \includegraphics[width=\textwidth]{k5/all_dynamics.pdf}
    \caption{\textbf{A} Transient length, attractor period, and their sum all 
    correlate with the entropy of the oiutput distribution. \textbf{B} Synergy 
    bias $B_{syn}$ correlates with the the sum of the attractor period and 
    transient length ($\rho=0.693, \; p<0.05$). \textbf{C} Effective
    connectivity, $k_e^*$ also correlates with the sum of atttractor period and
    transient length ($\rho=0.511, \; p<0.05$) and appears qualitatively quite
    similar to $B_{syn}$ in this regard.}
    \label{fig:dynamics}
\end{figure}

\begin{figure}
    \includegraphics{k5/ke_vs_bsyn_dyn.pdf}
    \caption{Synergy increases with effective connectivity}
    \label{fig:ke-bsyn}
\end{figure}

\begin{figure}
    \includegraphics{k5/bsyn_ke_reg.pdf}
\end{figure}

\begin{figure}
    \includegraphics{k5/ke_o_info_reg.pdf}
    \caption{$k_e^*$ correlates negatively with O-information}
    \label{fig:ke-oinfo}
\end{figure}

\begin{figure}
    \includegraphics{k5/directed_cana.pdf}
    \caption{Transfer entropy fails to identify input redundancy at the per input level}
    \label{fig:r-te}
\end{figure}

\begin{figure}
    \includegraphics{k5/corr_mat.pdf}
\end{figure}

\begin{figure}
\end{figure}

\end{document}