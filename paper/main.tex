\documentclass[12 pt]{article}
\usepackage[margin=1in]{geometry}

% Refs
\usepackage[style=nature, backend=bibtex]{biblatex}
\addbibresource{main.bib}
\usepackage[usenames,dvipsnames,svgnames,table]{xcolor}

% Math
\usepackage{amsmath}
\usepackage{amssymb}

% Figures
\usepackage{graphicx}
\newcommand{\figref}[1]{Fig.~\ref{fig:#1}}
\graphicspath{{../plots/}}

% subfigures
\usepackage{caption}
\usepackage{subcaption}

% line spacing
\usepackage{setspace}
\doublespacing

% author note
\usepackage{xcolor}
\newcommand{\note}[1]{\textcolor{red}{[#1]}}

\title{Luis Paper}
\author{Pat Wall}
\date{\today}

\begin{document}

\maketitle

\begin{abstract}
    Heeeyyyyyy this is the abstract baby. ill write this last probably!!
\end{abstract}

\section{Introduction}

\section{Results}

\begin{enumerate}
    \item bsyn and ke show similar relationships to system dynamics ($\rho=0.693, \; p<0.05$) and ($\rho=0.511, \; p<0.05$) [figure in supplement?] 
    \item Output entropy and dynamics?? \figref{ent-dyn} [get statistics?]
    \item 
\end{enumerate}

\begin{figure} 
    \includegraphics{k5/bsyn_dyn.pdf}
    \caption{Synergy Bias measures transient length kind of}
    \label{fig:bsyn-dyn}
\end{figure}

\begin{figure} 
    \includegraphics{k5/ke_dynamics.pdf}
    \caption{Synergy Bias measures transient length kind of}
    \label{fig:ke-dyn}
\end{figure}

\begin{figure}
    \includegraphics{k5/entropy_dynamics.pdf}
    \caption{rule table output entropy predicts dynamics}
    \label{fig:ent-dyn}
\end{figure}

\begin{figure}
    \includegraphics{k5/ke_vs_bsyn.pdf}
    \caption{a fi}
    \label{fig:be-bsyn}
\end{figure}

\section{Discussion}

\end{document}